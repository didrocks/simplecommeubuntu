%\thispagestyle{empty}
\null\vspace{\stretch{2}}
{\centering\fontsize{14}{16}\selectfont{}Simple Comme Ubuntu est un livre libre du projet framabook.\\
Vous pouvez vous en procurer une version papier sur\\\url{http://framabook.org/ubuntu.html}}\par
{\fontsize{12}{14}\selectfont{}Description du projet \textbf{Framabook}. Également disponible sur \url{http://framabook.org} :\par
\textbf{Un projet original}\\
Se démarquant de l'édition classique, les framabooks sont dits «~livres libres~» parce qu'ils sont placés sous une licence qui permet au lecteur de disposer des mêmes libertés qu'un utilisateur de logiciels libres.\\
Nous incitons à commander les livres pour soutenir le projet mais, quel que soit votre choix, libre à vous, en accord avec la licence, d'utiliser, copier, modifer et distribuer leurs versions numériques, ou tout simplement de tester avant d'acheter ;-)\par
\textbf{Un projet culturel}\\
De par leur licence libre, les framabooks s'inscrivent dans cette culture des biens communs qui, à l'instar de Wikipédia, favorise la création, le partage, la diffusion et l'appropriation collective de la connaissance.\par
\textbf{Un projet économique}\\
Nous aimerions montrer que, contrairement à certaines idées reçues, proposer des livres sous licence libre n'est pas un frein à la réussite commerciale du projet.\\
Le «~pari du livre libre~» c'est non seulement réussir à créer une collection de qualité mais c'est aussi arriver à rendre le modèle économiquement viable.\\
C'est important pour les auteurs qui n'ont pas ménagé leurs temps et leurs efforts pour nous offrir des ouvrages d'excellentes tenues. C'est important pout notre partenaire éditeur qui, partie prenante de l'aventure, a tout fait pour tirer les prix vers le bas sans sacrifier à la qualité d'impression. C'est important enfin pour le projet en lui-même et peut-être aussi par extension pour tout projet qui hésiterait à adopter un tel modèle jugé a priori à risque.\par
\textbf{Un projet à soutenir}\\
Si notre approche particulière rencontre votre adhésion, nous vous invitons à nous soutenir en liant ce site, en évoquant ce projet autour de vous mais aussi et surtout en achetant la déclinaison physique des framabooks, à savoir les livres édités par InLibroVeritas.\\
Un achat susceptible de vous être directement utile dans la découverte et l'usage d'un logiciel libre donné, mais un achat que vous pouvez également offrir en cadeau à vos proches, faisant alors d'une pierre deux coups : diffuser du logiciel libre et nous aider dans notre action.\\
Ajoutons enfin que la majorité des auteurs ont choisi de reverser un pourcentage de leurs bénéfices aux équipes et associations du logiciel libre dont traite leur ouvrage, illustrant ainsi le fameux cercle vertueux de la culture libre.\par
\textbf{Un projet en mouvement}\\
Notre objectif est de faire connaître et diffuser le logiciel libre au plus large public. Nous n'ignorons pas que cela implique souvent de rompre avec certaines habitudes acquises avec d'autres logiciels non libres.\\
Quitter la messagerie Outlook Express, la suite bureautique MS Office ou Windows Vista pour les alternatives libres Thunderbird, OpenOffice.org et GNU/Linux Ubuntu n'est pas toujours chose aisée.\\
Les framabooks, accessibles, illustrés et détaillés, facilitent de telles migrations et participent de ce mouvement de bascule du logiciel propriétaire vers le logiciel libre.\par
\newpage
\textbf{Un projet réactif}\\
Les caractéristiques des framabooks (collaboration, licences libres, tirages limités) permettent de suivre au plus près les évolutions des logiciels et d'offrir au lecteur un contenu rapidement réactualisé.\\
C'est ainsi que, par des auteurs parfois différents de ceux d'origine, les deux premiers volumes sur Thunderbird et Ubuntu ont été mis à jour et réédités peu de temps après les changements de version des logiciels en question.\par
\textbf{Un projet en quête d'auteurs}\\
Vous souhaitez nous rejoindre et écrire un futur framabook accompagné par notre fine équipe ? Il suffit d'accepter le principe des licences libres et d'être motivé par un travail passionnant mais de longue haleine. Vous trouverez de plus amples informations sur notre forum mais vous pouvez aussi nous contacter directement.\\
Pour vous donner un ordre d'idées sachez qu'en octobre 2008 nous totalisions près de quatre mille exemplaires vendus.\\
Exemples de poste à pourvoir : Inkscape, Firefox, Scribus, Freemind, Amarok, Drupal, Mediawiki, etc.\par
\textbf{Un projet matérialisé par InLibroVeritas}\\
Les framabooks seraient restés virtuels sans le concours de Mathieu Pasquini, fondateur de la maison d'édition pas comme les autres InLibroVeritas.\\
Parmi les nombreux autres livres disponibles chez l'éditeur on notera l'ouvrage collectif Tribune Libre, ténors de l'informatique libre, le dernier Perline (avec Thierry Noisette) Vote électronique : les boîtes noires de la démocratie, le très actuel Internet et Création de Philippe Aigrain, L'urgence de la métamorphose de Laurence Baranski et Jacques Robin ou encore la réédition du Hold-Up planétaire, la face cachée de Microsoft de Roberto Di Cosmo.\par
\textbf{Un projet Framasoft}\\
S'appuyant sur une association éponyme, Framasoft est un réseau de sites web collaboratifs à géométrie variable dont le dénominateur commun est le logiciel libre et son état d'esprit. Il vise à faire connaitre et diffuser le logiciel libre au plus large public.\\
Lieu d'informations, d'actualités, d'échanges et de projets, Framasoft, de par la diversité et le dynamisme de son réseau, est aujourd'hui l'une des portes d'entrée francophones du logiciel libre. Sa communauté accompagne ceux qui souhaitent substituer leurs logiciels propriétaires par des logiciels libres. Elle attache une attention toute particulière au processus de migration du système d'exploitation Microsoft Windows, XP ou Vista, vers GNU/Linux.\\
Parmi les nombreuses ressources du réseau on trouve un annuaire de plus d'un millier de logiciels libres, une clé USB dite Framakey spécialisée dans les applications portables libres Windows, un forum où le néophyte est le bienvenu et un blog qui constate au quotidien que le logiciel libre est en train de libérer bien plus que du simple code et Framabook est bien placé pour en témoigner !
}
