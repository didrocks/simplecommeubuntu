\TitreIntro{Conventions utilisées dans ce livre}
\label{RefExemple}
Pour permettre une lecture et un repérage plus simple de ce livre, voici les conventions que l'on s'est fixé :\par
Un renvoi vers une autre partie du livre est indiqué de la sorte : cf. page \pageref{RefExemple}.\par
Les notes de bas de page sont représentées ainsi\NotePage{Je ne sais pas si vous avez remarqué, mais la référence précédente était une référence auto-récursive ;)}.\par
La navigation entre un menu et un sous-menu est séparée par cette flèche \FlecheDroite très esthétique !\par
Des chemins vers des fichiers et des dossiers sont présentés de cette manière : \Chemin{vers/l/infini/et/au/delà}.\par
Une touche du clavier est mise en évidence comme la touche \Touche{Entrée}. Un signe + est ajouté si vous devez presser simultanément plusieurs touches.\par
Un élément décrit dans le glossaire$^*$ bénéficie également d'une mise en évidence particulière.\par
Les commandes à entrer dans un terminal sont mises en valeur de cette manière : \Commande{Une ligne de commande}. Elles sont -- le plus souvent -- à saisir sur une seule ligne, la mise en page d'un livre ne permettant pas, la plupart du temps, de l'écrire ainsi.\par
Une note plus longue -- et plus importante -- est indiquée de la sorte : \begin{nota}Cette note devrait normalement apporter des précisions supplémentaires au sujet précédent. Cependant, ces informations ne sont pas non plus capitales.\end{nota}\par
Les notes plus importantes, quant à elles, sont représentées comme ceci : \begin{attention}Ceci est une note importante, elle essaie d'attirer votre attention sur un point précis à ne pas négliger afin que la suite des opérations indiquées dans le livre se déroule sans heurts.\end{attention}\par
Les lignes de code représentent la sortie du terminal. Cette présentation est également utilisée lorsqu'un grand nombre d'éléments sont à entrer dans un fichier : \Code{\# Ceci est un commentaire dans un fichier de configuration\\
\# Le plus souvent, ce dernier est présent sur plusieurs lignes}\par
Quelques citations jalonnent ce livre et sont représentées de la sorte :\par
\begin{citationlongue}{Pythagore a dit}
Les amis sont des compagnons de voyage, qui nous aident à avancer sur le chemin d'une vie plus heureuse.
\end{citationlongue}\par
Ce livre existe également dans un format électronique où les adresses internet comme celle de framabook -- \url{http://www.framabook.org} -- sont des hyperliens, et les liens internes comme les notes de bas de page sont également actifs.\par
\vspace{\stretch{1}}
Je tiens enfin à rappeler que les adresses Internet sont malheureusement par nature, appelées à varier : un lien, valide lors de l'écriture de ce livre, peut donc devenir invalide.
\vspace{\stretch{10}}
