\TitreIntro{À propos de l'auteur}
En 1990, \Personne{Didier}{Roche} se découvre très jeune une grande passion pour l'informatique. Il s'adonne très vite, avec un réel intérêt, à la programmation et en apprend de très nombreux langages dans ses années de collège-lycée. À partir de 2002, il effectue des études supérieures d'ingénieur généraliste à l'ECAM\NotePage{École Catholique d'Arts et Métiers, Lyon}. Profitant de ces années pour entrer dans une association humanitaire dont l'objectif est d'amener des ordinateurs dans les écoles et universités d'Afrique et de former sur place étudiants et professeurs à l'utilisation de l'outil informatique -- association Afric'Edu\NotePage{Association pour la Formation de Réseaux Internet Commis à l'Education et au Développement des Universités} basée à Lyon -- ceci le conduira notamment au Togo pendant le mois de juillet 2006, donnant accès à l'informatique à plus de 2000 élèves et professeurs. Actuellement, il travaille chez Canonical qui est la société qui sponsorise Ubuntu en tant que membre de la «~Desktop team~», responsable de l'environnement graphique par défaut d'Ubuntu. Il s'occupe également de la version netbook d'Ubuntu, Ubuntu Netbook Edition (UNE, anciennement UNR).\par
Concernant sa pratique de GNU/Linux, son premier essai de migration date de 1996 avec une Red Hat qui s'avèrera être un échec puisqu'il n'y resta pas très longtemps. Sa seconde migration, cette fois réussie à l'aide d'une Mandrake\NotePage{Nommée aujourd'hui Mandriva pour une sombre histoire de licence avec \ldots{} le magicien du même nom !} 7, fit de lui un Linuxien convaincu ! En 2004, il découvrit la distribution Ubuntu\NotePage{Alors qu'elle n'avait pas encore de nom définitif !} qui allait devenir son système d'exploitation principal. Il se découvre alors une réelle passion pour le Logiciel Libre et y consacre la plupart de son temps «~libre~». Il est aujourd'hui le secrétaire de l'association francophone des utilisateurs d'Ubuntu, \url{http://www.ubuntu-fr.org}, participe activement au développement de la distribution Ubuntu en tant que Core Développeur (développeur officiel de la distribution) et membre du comité de pilotage des Framabooks.\par
Didier Roche a choisi de reverser 20 \% de ses droits d'auteur également répartis entre les associations Ubuntu-fr et Framasoft.org afin de les soutenir et de les remercier pour leurs extraordinaires travaux.\par
