\TitreIntro{Remerciements}
\begin{Remerciements}
Je tiens à remercier tout d'abord les développeurs de logiciels libres et leurs contributeurs, en particulier ceux de la Fondation Ubuntu, pour leur sens de l'intérêt général, du partage, de l'entraide et de l'innovation. Merci également à \Personne{Mark}{Shuttleworth} pour son dynamisme et sa communication sur les objectifs de sa distribution.\par
J'entretiens une sincère pensée à ma compagne qui est à mes côtés et supporte mes (trop) nombreuses heures passées à contribuer à l'écosystème du Libre. Je n'oublie pas non plus mes parents qui m'ont permis d'accéder à l'outil informatique dès mon plus jeune âge. Je tiens également à saluer \Personne{Vanessa et Dimitri}{Perrin} pour leur soutien et pour avoir rendu mon séjour en Irlande agréable.\par
Cette tâche a été effectuée à partir de l'énorme travail d'\Personne{Anthony}{Carré}\NotePage{\url{http://yeknan.free.fr}}, initialement adapté au format imprimable par \Personne{Joseph}{Maillarde}\NotePage{\url{http://wenux.net}}, ce qui m'a plus que motivé à la rédaction de ce document. J'aimerais aussi inclure dans les remerciements les trois co-auteurs \Personne{Julien}{Rottenberg}, \Personne{Guillaume}{Ludwig} et \Personne{Joseph}{Massot}. Leur livre n'a malheureusement jamais pu sortir. Les parties sur l'installation de l'imprimante et du scanner sont (quasi) intégralement basées sur leur travail, ainsi que quelques ajouts dans la description des menus.\par
Cette documentation est également «~librement~» inspirée de quelques billets du blog de \Personne{David}{Szerman}\NotePage{\url{http://www.szdavid.com}}, de \Personne{Jean-Baptiste}{Hétier}\NotePage{\url{http://www.think-underground.com}}, de Asher\NotePage{\url{http://asher256.tuxfamily.org}} -- par son travail sur son dépôt -- de  \Personne{Grégory}{Gutierez}\NotePage{\url{http://petitlinux.greguti.com}}, sans oublier le wiki de notre chère communauté francophone\NotePage{\url{http://doc.ubuntu-fr.org}} et sa mailing-list toujours aussi accueillante. L'excellent site \url{http://www.whylinuxisbetter.net} a également été une source d'inspiration importante, merci à  \Personne{Manu}{Cornet}, son créateur.\par
Ce livre réalisé en \LaTeX{} n'aurait jamais été possible sans l'excellent guide «~Tout ce que vous avez voulu toujours savoir sur \LaTeX{} sans jamais oser le demander~» de \Personne{Vincent}{Lozano}, autre livre de la collection Framabook. Consultez \url{http://www.framabook.org} !\par
Un grand merci également à toutes les personnes qui ont apporté leur pierre à l'édifice par le biais du forum, notamment \Personne{Raphaël}{Bourven} pour son énorme travail de relecture\NotePage{Et il a même re-signé pour les nouvelles versions, le bougre !} et les nombreuses corrections apportées par la communauté d'Ubuntu-fr dont \Personne{Fabrice}{Braillon}, \Personne{Franck}{Chadel}, \Personne{Quentin}{Bricard} et \Personne{Claude}{Crozet}, ainsi que par le contributeur très productif qui sait se lancer dans un sprint final éhonté\NotePage{Il ne sait que trop bien ce que signifie «~On boucle demain midi !~» ;-)}, à de multiples reprises : \Personne{Bruno}{Le Clainche}.\par
Cette version a également bénéficié du groupe de lecture de Framabook par les participations actives de \Personne{Barbara}{Bourdelles}, \Personne{Yann}{Gueron}, \Personne{Raymond}{Rochedieu}, \Personne{Antoine}{Blanche}, Metta Om ainsi que de Kinouchou.\par
Un grand merci notamment au TdCT sur le forum ubuntu-fr et à quelques contributeurs de la documentation ubuntu-fr qui ont grandement participé au rafraîchissement des images de ce livre après mon appel à l'aide. Ont participé à cet effort : DisasteR, le n@nyl@nd, rmy, YannUbuntu, Huats, tshirtman, ArzhurBZH, kankan\_01 et ma chère Julie.\par
Enfin, je témoigne ma gratitude à \Personne{Alexis}{Kauffmann}\NotePage{\url{http://framablog.org}}, fondateur de Framasoft\NotePage{\url{http://www.framasoft.net}} pour avoir cru qu'il était possible de transformer ma documentation en un livre, qui, je l'espère, vous plaira, ainsi qu'à \Personne{Mathieu}{Pasquini}\NotePage{\url{http://www.inlibroveritas.net}}, mon éditeur, pour son soutien même dans les moments les plus difficiles.\par
Ce livre essaie de respecter au maximum les règles de la typographie française, même si\NotePage{Oui, je fais du préventif pour éviter toute critique !}, je suis tout à fait conscient que, comme je vous le présenterai en \ref{refMajAccent}, la première lettre de l'introduction du chapitre \ref{RefDebutChap7} page \pageref{RefDebutChap7} devrait être accentuée. Il s'agit toutefois d'un problème auquel j'ai fait face avec \LaTeX{} sans, malheureusement, arriver à le résoudre à cet instant.\par
Pour toute remarque ou suggestion constructive concernant ce livre, vous pouvez ouvrir un sujet dans la section «~Framabook.org~» de Framagora\NotePage{Forum de Framasoft} à l'adresse suivante : \url{http://forum.framasoft.org}.\par
\end{Remerciements}
