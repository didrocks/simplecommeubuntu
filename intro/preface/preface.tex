\TitreIntro{Préface de l'auteur}
Beaucoup d'utilisateurs sont mécontents des services offerts par leur \RefGlossaire{Système}{OS}{Système d'exploitation}{système d'exploitation} actuel : bugs, plantages fréquents, travail effacé ou perdu et formatages récurrents. Ils entendent alors parler de Linux, mais dans leur esprit, ce dernier reste un \RefGlossaire{Système}{Système d'exploitation}{OS}{OS} complexe, où tout doit se faire «~à la main~». De plus, certains utilisateurs reviennent sous Windows après un court passage à Linux en expliquant que «~c'est compliqué, c'est vraiment un \RefGlossaire{Système}{OS}{Système d'exploitation}{système d'exploitation} qui n'est pas fait pour les utilisateurs mais pour les programmeurs~» (quand ils ne disent pas «~programmateurs~» ! :-)), ce qui ajoute encore plus à cette idée largement répandue d'inaccessible complexité.\par
Alors, GNU/Linux est-il réellement un \RefGlossaire{Système}{OS}{Système d'exploitation}{système d'exploitation} élitiste ? Êtes-vous vraiment obligés de passer des heures et des heures pour configurer correctement votre \RefGlossaire{Système}{OS}{Système d'exploitation}{système d'exploitation}, installer un logiciel et autres opérations qui, je le vois déjà, ne vous réjouissent pas ?\par
Le principal frein concernant ce point est assurément la multitude des documentations existantes sur l'\RefGlossaire{-}{L'Internet}{Internet}{Internet}, ce qui peut faire perdre de vue à l'utilisateur débutant les clefs vraiment essentielles à une utilisation quotidienne. Un deuxième problème, indépendant de Linux lui-même, est que «~GNU/Linux n'est pas Windows~» : l'attachement du nouveau venu à ses habitudes «~fenêtresques~» est, en effet, assez tenace. Je prendrai comme exemple le téléchargement d'un programme depuis l'\RefGlossaire{-}{L'Internet}{Internet}{Internet}. On va sur un site de gratuiciels le plus souvent, et on prend un fichier archive (souvent un .zip). On s'attend à trouver un fichier .exe à l'intérieur de ce dernier et à double-cliquer dessus pour l'installer\NotePage{Puis d'effectuer «~suivant~», «~suivant~», «~suivant~», sans lire la licence, évidemment\ldots{}}. Le nouvel utilisateur linuxien va essayer de reproduire le même comportement sur son nouvel OS : il va télécharger une archive\NotePage{Le plus souvent .tar.gz} sur un site, mais ne va pas trouver son «~si confortable~» setup.exe ! Il cherche sur un forum et on lui parle alors de fichier «~source~» à «~compiler~». Et voilà que des utilisateurs, venant d'installer Linux pour la première fois, essaient, dans la foulée, de compiler un logiciel. Par la force des choses, ces débutants vont rapidement arriver au constat que GNU/Linux n'est vraiment pas «~user-friendly~».\par
Pourquoi cet essai systématique de mimétisme ? La plus simple des réponses trouve très certainement son explication dans l'existence d'un facteur «~confort\NotePage{Et sûrement de facilité !}~», car lorsque l'on maîtrise un \RefGlossaire{Système}{OS}{Système d'exploitation}{système d'exploitation}, on s'attend à retrouver ses repères, puisque «~on nous a appris comme cela~». Et si on peine pour une chose aussi bête que l'installation d'un logiciel, on se dit que le système n'est pas adapté. Cependant, si vous vous rappelez vos premiers pas en informatique, rien n'a été inné : c'est vous qui vous êtes adapté à votre \RefGlossaire{Système}{OS}{Système d'exploitation}{système d'exploitation} et en avez pris les -- parfois mauvaises -- habitudes. Ne vous attendez pas à maîtriser Linux en une journée ; est-ce le temps qu'il vous a fallu pour maîtriser votre \RefGlossaire{Système}{OS}{Système d'exploitation}{système d'exploitation} actuel ?\par
Il faut donc «~penser autrement~» car vous êtes sur un \RefGlossaire{Système}{OS}{Système d'exploitation}{système d'exploitation} différent, dont la philosophie n'est pas de copier le fonctionnement d'autres systèmes, mais d'en offrir une alternative avec d'autres modes de pensée\NotePage{Vous verrez par exemple au chapitre 4 qu'installer une application est vraiment une sinécure sous Ubuntu.}. Mais pour cela, me direz-vous, il faut que l'on vous aide et vous montre la voie pour être efficace le plus rapidement possible. C'est en ce sens que ce livre a été écrit et vous verrez que Linux n'est pas plus compliqué que Windows\NotePage{Sur certains points, il est même beaucoup plus simple !}, mais correspond à une logique «~différente~». Les mauvaises migrations -- les personnes revenues à Windows -- sont principalement des individus autodidactes qui maîtrisaient très bien leur \RefGlossaire{Système}{OS}{Système d'exploitation}{système d'exploitation}, mais étaient totalement incapables de s'adapter à Linux car elles en attendaient exactement le même comportement. Bouleverser ses habitudes ne vient pas sans heurts, en effet, mais lorsque l'on y arrive, quelle satisfaction !\par
Je peux vous certifier que j'utilise quotidiennement GNU/Linux. Eh bien, je peine parfois énormément sous Windows à effectuer une opération pourtant simple, alors que je pense avoir bien maîtrisé ce \RefGlossaire{Système}{OS}{Système d'exploitation}{système d'exploitation} à une époque. Selon moi, la distribution Ubuntu, que je suis, à ma grande satisfaction, depuis sa création -- lorsque la première mouture n'était pas encore véritablement sortie -- jusqu'à aujourd'hui, apporte vraiment une alternative simple à Windows. Bien évidemment, ce jugement n'engage que moi et, bien qu'il existe de très nombreuses distributions GNU/Linux ou encore Mac OS, les moyens mis en œuvre dans celle-ci font preuve d'un véritable professionnalisme. Vous l'aurez compris, Ubuntu est clairement orientée vers les utilisateurs et les entreprises.\par
À partir du travail d'un blogueur de la communauté d'Ubuntu, «~Yekcim~», qui décrivait l'installation de cet \RefGlossaire{Système}{OS}{Système d'exploitation}{OS} et le listing de quelques programmes et jeux, j'ai pleinement pris conscience de la nécessité d'une documentation française aisément identifiable qui guide les utilisateurs débutants dans leurs premiers pas sous ce \RefGlossaire{Système}{OS}{Système d'exploitation}{système d'exploitation}. J'ai donc repris son ensemble de billets\NotePage{La licence le permettait et est la même que celle de ce livre} qui s'étendait alors sur une quinzaine de pages pour réaliser une documentation très expurgée, qui devait rester la plus succincte possible. Puis, de fil en aiguille, je me suis pris au jeu en ajoutant logiciels, jeux, astuces d'utilisation, précisions sur le système\ldots{}\par
La documentation a connu, à ma grande surprise, un vrai succès sur le forum de la communauté francophone d'Ubuntu et de plus en plus de personnes l'ont conseillée aux débutants. De plus, elle a été adaptée par des utilisateurs vers des distributions comme Kubuntu et Xubuntu ! J'ai ensuite été contacté par Framasoft qui m'a présenté son projet de collection de livres sous \AddIndex{Philosophie}{-}{Licence Libre}{licence libre}. J'y ai donc pris part et cela m'a motivé pour améliorer et finaliser ce qui n'était encore, à l'époque, qu'une simple documentation en ce véritable livre que vous tenez entre vos mains.\par
À la sortie de la version Feisty Fawn d'Ubuntu, une mise à jour était nécessaire aux vues des nouveautés apportées par cette version. J'apprenais, pour le milieu professionnel, les rudiments de \LaTeX{}. J'ai alors tout naturellement proposé à Framabook que l'on uniformise les présentations des différents livres en préparation en passant à ce format, proposition ayant été couronnée de succès après débats internes et externes. Une grande refonte a donc été effectuée afin de changer la présentation, et de la rendre plus conforme aux standards du livre. La version que vous tenez entre les mains est le résultat de cette présentation en \LaTeX{}, mise à jour et adaptée à la version Maverick Meerkat (10.10).\par
Je tiens enfin à souligner que cette documentation n'est vraiment pas une confrontation «~GNU/Linux versus Windows~» en faisant tout pour enjoliver le premier et détruire le second, bien que quelques piques et réécritures orthographiques intentionnelles je l'admets, existent au cours des divers chapitres. Je vous présente un nouveau \RefGlossaire{Système}{OS}{Système d'exploitation}{système d'exploitation} et vous explique ce qui le différencie de Windows.\par
Croyant à l'expression «~Talk to the mind, not to the brain~»\NotePage{Parlez à l'esprit et non au cerveau}, ce livre est écrit dans un style plutôt «~libre~» -- décidément ! -- afin de ne pas rendre le résultat trop indigeste. J'espère ainsi que chaque chapitre vous donnera un peu plus envie d'aller de l'avant.\par
