\TitreIntro{Préface}
Quand je croise quelqu'un qui n'a pas encore vu une série comme Les Soprano ou Battlestar Gallactica, au lieu de hurler au scandale et à l'inculture crasse (je pourrais, notez), je ne peux pas m'empêcher de penser «~petit(e) veinard(e)~». Parce que le plaisir de la découverte ne m'est plus accessible. Je peux les revoir, certes, mais il m'est difficile aujourd'hui de m'extasier devant tel retournement ou d'être assommé par telle révélation sur un personnage. La première fois n'arrive qu'une fois. C'est bien dommage.\par
Donc si vous être en train de lire cet (excellent) ouvrage parce que vous avez décidé de franchir le pas, de partir à la découverte d'Ubuntu\ldots{} petits veinards. Je suis un peu jaloux. Comme pour une série ou un film, je m'en voudrais de dévoiler dès maintenant toutes les petites découvertes qui vous attendent. Je ne vais pas «~spoiler~» Ubuntu, rassurez-vous, mais les premiers pas dans ce système libre vous réservent quelques bonnes surprises, et quelques retournements de situation sympathiques. Mais je ne vais pas jouer au vieux sage qui regarde d'un œil malicieux le jeune disciple qui a décidé de se lancer sur la voie du shaolin, je n'en suis moi-même qu'à mes débuts. Un peu plus de trois mois à l'heure où j'écris ces lignes que j'utilise quotidiennement un ordinateur portable sous Ubuntu. Et mises à part quelques mésaventures sans grande importance que j'ai racontées ailleurs\NotePage{Le journal d'un novice sur Ecrans.fr, \url{http://www.ecrans.fr/+-le-journal-d-un-novice-+.html?page=journal}}, tout roule parfaitement bien. J'ai même réussi à faire fonctionner ma souris bluetooth, c'est dire !\par
Ces quelques mois de pratique ne font pas de moi un gourou d'Ubuntu, loin s'en faut. Et je n'ai pas la prétention de le devenir un jour. Mais ma petite expérience m'a appris quelques petits trucs que je me permets de retranscrire sous la forme de petites règles. Rien d'impératif, comme toutes les règles, elles existent aussi pour être transgressées.\par
\begin{description}
\item[Règle N°1 :] La découverte d'Ubuntu, c'est un jeu. Non pas un jeu d'enfant (quoique\ldots{}), mais un parcours ludique qu'il faut aborder l'esprit ouvert. Il y aura sans doute des obstacles, des petits («~hum\ldots{} dois-je cliquer sur «~Appliquer~» ou sur «~Annuler~») et des plus grands («~comment puis-je faire pour utiliser cette imprimante qui date du 20e siècle ?~»), mais ils ne sont pratiquement jamais insurmontables. Et si on prend la chose du côté ludique, on s'amuse plus qu'on ne rouspète. En plus, pour reprendre une expression du jeu vidéo, si la difficulté devient trop grande, les soluces sont sur Internet.
\item[Règle N°2 :] Installer Ubuntu, c'est un choix. Faire le choix de son système d'exploitation est sans doute une nouveauté pour vous. En tout cas, ça l'a été pour moi. Quinze ans sous Windows avant de franchir le pas. C'est la seule fois où, à l'achat d'un ordinateur, je me suis demandé quel système j'allais prendre. Jusqu'ici, l'alternative consistait à acheter une machine Apple livrée avec son propre système, ou un PC plus classique et donc Windows. Sans avoir de choix, car on ne se pose même pas la question. Pour passer à Linux, il faut que cette question existe. Avec ce choix, il n'y a aucune obligation en absolu. Vous l'avez peut-être fait tout simplement à cause de la gratuité du système. Mais même dans ce cas, je ne saurais trop vous conseiller de vous renseigner sur ce qui a permis que vous puissiez le faire. Logiciel libre, GNU, GPL, open-source, l'histoire de ce dernier quart de siècle est passionnante. Aussi, prendre conscience de la somme de travail qui a permis à Ubuntu et à d'autres systèmes et logiciels de voir le jour permet de relativiser ses petits soucis de souris Bluetooth (non, je ne fais pas une fixation).
\item[Règle N°3 :] Vous n'êtes pas seuls. C'est sans doute un des points les plus extraordinaires de Linux et donc d'Ubuntu : la serviabilité de ses utilisateurs. Si vous avez un problème, non seulement vous pouvez être sûr que quelqu'un l'a déjà eu et que la réponse se trouve dans un des nombreux forums de discussions et blogs existants, mais si par le plus grand des hasards ce n'était pas le cas, il suffit de poser la question pour qu'un habitué vienne à votre secours. Bon, d'accord, parfois, ils sont un peu bizarres et ils parlent un langage qui peut paraître étrange, mais croyez-moi, ils feront tout pour vous aider. Pourquoi ? Peut-être à cause des deux premières règles que j'ai énoncées. Et vous verrez, d'ici quelques mois, vous finirez peut-être par aider un débutant à se dépatouiller. C'est toujours très gratifiant.
\item[Règle N°4 :] Ubuntu, ce n'est qu'un système d'exploitation. Rien de plus. Je sais, c'est déjà énorme, mais quand vous allumez votre ordinateur, ce n'est pas pour admirer l'écran de lancement. Enfin, on peut, mais ça devient un peu lassant à la longue. Le but d'un système d'exploitation, c'est de se faire oublier. D'ici quelques semaines (quelques jours, même), vous allumerez votre ordinateur, vous lancerez votre navigateur web, vous retoucherez vos photos, vous écrirez des textes sans même vous soucier de ce qui fait tourner tout ça. Aujourd'hui, des milliers de personnes, moi le premier, utilisent Ubuntu pour une utilisation courante (Internet, bureautique, images, etc.) sans rencontrer le moindre problème. Aucune raison qu'il en soit autrement dans votre cas.
\item[Règle N°5 :] Il est interdit de parler du Fight Club. Euh\ldots{} Non, ça, c'est une autre histoire.
\end{description}
Mais c'est vrai, faire ses premiers pas sur un nouveau système d'exploitation, c'est parfois un peu intimidant. C'est un peu comme partir en voyage dans un pays qu'on ne connaît pas et dont on ne parle pas la langue (mais on a vu des photos, il paraît que c'est très joli!). C'est là que Simple comme Ubuntu entre en jeu. Le livre de \Personne{Didier}{Roche}, c'est un peu le Guide du Routard d'Ubuntu. On y trouve un descriptif complet de l'endroit, des itinéraires conseillés, des bonnes adresses, les bons plans et les lieux à visiter (ne ratez pas le gestionnaire de paquets, c'est magnifique). En suivant ses indications, difficile de se perdre. Et vous êtes sûr de ne rien rater d'important. Et puis, peut-être, après avoir écorné les pages, souligné les petits trucs, cerclé les points importants (rien que pour ça, achetez le livre, en pdf, c'est plus dur), vous vous sentirez à l'aise pour partir à l'aventure sur les chemins de traverse d'Ubuntu.\par
Mais n'allons pas trop vite. Chaque chose en son temps. Je ne veux pas vous retarder, la séance va commencer. Prenez vos places, installez-vous confortablement.\par
Il était une fois un koala avec du karma\ldots{}\par
{\vspace{\stretch{1}}{\centering%
		{\fontsize{12}{36}\selectfont \Personne{Erwan}{CARIO}}\par Journaliste sur Écrans.fr\par\vspace{\stretch{4}}}%
