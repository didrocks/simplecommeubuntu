\chapitre{Et maintenant ?}{Félicitations! Vous voici intronisé}{ jeune Ubuntero !! Vous êtes maintenant en possession d'une bonne distribution GNU/Linux : vous en maîtrisez le fonctionnement et, cette utilisation sera certainement amplement suffisante pour vos tâches quotidiennes ! Toutefois, vous avez peut-être envie d'en savoir plus sur la philosophie du \AddIndex{Philosophie}{-}{Libre}{Libre} développée dans le premier chapitre et, même vouloir contribuer, qui sait ?! Voici une très brève introduction -- un comble pour un chapitre de conclusion ! -- afin de vous mettre le pied à l'étrier, et savoir où chercher les informations.}
Si vous avez soif de connaissance, désirez personnaliser davantage votre système -- par exemple, vous pouvez installer un GRUB beaucoup plus joli que celui proposé par défaut grâce à grub-gfx, installer des applets supplémentaires\ldots{} -- si vous avez un problème, n'oubliez pas pour autant le magnifique centre d'aide Ubuntu intégré au système\NotePage{Qui n'a rien à voir en terme de clarté et d'utilité à celui d'un autre \RefGlossaire{Système}{Système d'exploitation}{OS}{OS} que vous avez sûrement ouvert une seule et unique fois avant de le refermer aussi vite ;-)}, accessible par Système \FlecheDroite Aide et soutien. Vos recherches seront à coup sûr pertinentes, mais vous pouvez également vous laisser guider par les rubriques particulièrement bien organisées.\par
Si vous ne trouvez pas de réponse, ou décidez tout simplement de vous impliquer un peu plus, n'hésitez pas à vous joindre à la communauté Ubuntu ! Le site de référence de la communauté Ubuntu francophone est \url{http://www.ubuntu-fr.org}. Vous y trouverez notamment, un forum très accueillant, une documentation sous forme de Wiki, et aussi un Planet regroupant plusieurs blogs concernant Ubuntu !\par
Servez-vous de la documentation pour fignoler les étapes qui auraient pu poser problème --- par exemple, chez certaines personnes, les partitions Windows ne sont pas automatiquement disponibles, chez d'autres, le \RefGlossaire{Réseau}{Connexion sans fil}{Wi-Fi}{Wi-Fi} n'est pas activé\ldots{} Il y a aura presque toujours une solution pour vous. Si tel n'est pas le cas, posez, après avoir effectué une recherche sur Google par exemple, votre question sur le forum. La communauté vous accueillera à bras ouverts, c'est certain.\par
Le temps et l'envie aidant, vous pourrez, vous aussi, contribuer, en assistant les autres utilisateurs ou en écrivant des articles sur le Wiki, en suggérant de nouvelles fonctionnalités ou encore en traduisant des applications en français ! Tout est possible ! Et, encore une fois, tous les renseignements nécessaires se trouvent sur la documentation du site Ubuntu-fr. N'hésitez pas également à participer aux Ubuntu Party à travers l'espace francophone, cf \url{http://www.ubuntu-party.org}.\par
N'oubliez pas non plus de vous informer sur l'excellente encyclopédie \AddIndex{Philosophie}{-}{Libre}{Libre} en ligne : \url{http://fr.wikipedia.org}. Si vous souhaitez en connaître un peu plus sur le \AddIndex{Philosophie}{-}{Libre}{Libre} et sa philosophie, direction LE site du \AddIndex{Philosophie}{-}{Libre}{Libre} : \url{http://www.framasoft.net}.\par
\vspace{\stretch{1}}
\begin{center}
À bientôt et profitez bien de votre système Ubuntu !\\Vous êtes maintenant \textbf{Libre} !\par
\end{center}
\vspace{\stretch{2}}
