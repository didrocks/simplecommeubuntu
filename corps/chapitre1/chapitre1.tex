\def\NbMembresUbuntufr{170 000}
\chapitre{Introduction}{Avant d'entrer dans le vif du sujet,}{ une présentation d'Ubuntu Linux  et de la philosophie du \AddIndex{Philosophie}{-}{Libre}{Libre} en général peut sembler nécessaire. En effet, malgré la médiatisation grandissante du mouvement du \AddIndex{Philosophie}{-}{Libre}{Libre}, de -- trop -- nombreuses personnes assimilent le \AddIndex{Philosophie}{-}{Libre}{Libre} à la gratuité. Si vous pensez encore que ces deux notions sont équivalentes, vous verrez qu'à la fin de la lecture de ce chapitre, votre avis aura changé et vous mesurerez plus précisément les différences entre \AddIndex{Philosophie}{-}{Libre}{Libre} et \RefGlossaire{Philosophie}{-}{Logiciel Propriétaire}{propriétaire}, ainsi que les enjeux qui en découlent.}
\section{Qu'est-ce que le mouvement GNU ?}
En 1984, \AddIndex{Philosophie}{-}{Richard Stallman}{\Personne{Richard Matthew}{Stallman}}\NotePage{Également connu sous le diminutif RMS}, chercheur en informatique du MIT\NotePage{Institut de Technologie du Massachussetts} quitte son poste et se consacre à l'écriture d'un \RefGlossaire{Système}{OS}{Système d'exploitation}{système d'exploitation} \AddIndex{Philosophie}{-}{Libre}{Libre} du nom de GNU\NotePage{Acronyme récursif de GNU is Not Unix}. Il annonce l'année suivante la création de la \AddIndex{Philosophie}{-}{Free Software Foundation}{FSF}\NotePage{Free Software Foundation, \url{http://www.fsf.org/}} afin de supporter ce projet.\\
C'est durant ces années qu'il écrit ce qui deviendra les préceptes du \RefGlossaire{Philosophie}{Libre}{Logiciel Libre}{Logiciel Libre}. La concrétisation en est la publication en 1989 de la première version de la licence GPL\NotePage{General Public Licence} qui sera alors le fondement éthique, juridique et politique du mouvement du \AddIndex{Philosophie}{-}{Libre}{Libre}.\par
Vous trouverez plus d'informations sur le mouvement GNU sur le site \url{http://www.gnu.org}.
\section{Qu'est-ce qu'un logiciel libre ?}
L'expression «~\RefGlossaire{Philosophie}{Libre}{Logiciel Libre}{Logiciel Libre}~» fait référence à la liberté et non pas au prix. Pour comprendre le concept, vous devez penser à la «~liberté d'expression~», pas à «~l'entrée libre~».\par
Le \RefGlossaire{Philosophie}{Libre}{Logiciel Libre}{Logiciel Libre} est fondé sur une notion de liberté, contrairement au \RefGlossaire{Philosophie}{Propriétaire}{Logiciel Propriétaire}{Logiciel Propriétaire} qui vous accorde une licence seulement -- vous n'êtes donc jamais propriétaire de votre logiciel, mais n'avez qu'une licence d'utilisation accordée par l'éditeur -- où l'on vous dit, a contrario, ce que vous n'avez pas le droit de faire.
L'expression «~\RefGlossaire{Philosophie}{Libre}{Logiciel Libre}{Logiciel Libre}~» fait référence à la liberté pour les utilisateurs d'exécuter, de copier, de distribuer, d'étudier, de modifier et d'améliorer le logiciel. Plus précisément, elle fait référence à quatre types de liberté pour l'utilisateur du logiciel :
\begin{description}
\item[Liberté 0] La liberté d'exécuter le programme, pour tous les usages.
\item[Liberté 1] La liberté d'étudier le fonctionnement du programme, et de l'adapter à vos besoins. Pour ceci l'accès au code source est une condition requise.
\item[Liberté 2] La liberté de redistribuer des copies, donc d'aider votre voisin.
\item[Liberté 3] La liberté d'améliorer le programme et de publier vos améliorations, pour en faire profiter toute la communauté. Pour ce faire, l'accès au code source est une condition requise.
\end{description}
Un programme est un \RefGlossaire{Philosophie}{Libre}{Logiciel Libre}{Logiciel Libre} si les utilisateurs ont toutes ces libertés\NotePage{Oui, le fait que la première liberté ait le numéro 0, c'est très \RefGlossaire{Jargon}{-}{Geek}{geek}}. Ainsi, vous êtes \AddIndex{Philosophie}{-}{Libre}{Libre} de redistribuer des copies, avec ou sans modification, gratuitement ou non, à tout le monde, partout. Être \AddIndex{Philosophie}{-}{Libre}{Libre} de faire ceci signifie -- entre autres -- que vous n'avez pas à demander ou à payer pour en avoir la permission. Cela permet de garantir la Liberté -- savoir ce qui se passe sur votre ordinateur, pouvoir changer de système aisément par l'utilisation de formats ouverts --, l'Égalité -- avoir accès à un logiciel à un prix bas ou gratuitement --, et la Fraternité -- avoir le droit de redistribuer légalement ses logiciels à ses amis.\par
Vous devez aussi avoir la liberté de faire des modifications et de les utiliser à titre personnel dans votre travail ou vos loisirs, sans en mentionner l'existence. Si vous publiez vos modifications, rien ne vous oblige à prévenir quelqu'un en particulier ou à le faire d'une manière ou d'une autre.\par
La liberté d'utiliser un programme est la liberté pour tout type de personne ou d'organisation de l'utiliser pour tout type de système informatique, pour tout type de tâche et sans être obligé de communiquer ultérieurement avec le développeur ou tout autre entité spécifique.\par
Si vous souhaitez plus d'informations sur les \RefGlossaire{Philosophie}{Libre}{Logiciel Libre}{logiciels libres}, l'adresse\NotePage{D'où ce texte est tiré} est la suivante : \url{http://www.gnu.org/philosophy/free-sw.fr.html}.
\newpage
\section{Pourquoi installer GNU/Linux ?}
\ImgPar{r}{2.82}{corps/chapitre1/img/Tux.png}
Le projet GNU arrive en 1991 avec de très nombreux outils \AddIndex{Philosophie}{-}{Libre}{libres}, mais il lui manque un élément central : le \RefGlossaire{Système}{Kernel}{Noyau}{noyau}. Cet élément est essentiel car il gère la mémoire, le microprocesseur, les périphériques comme le clavier, la souris, les disques durs\ldots{}\\
C'est à cette époque qu'un étudiant finlandais, \Personne{Linus}{Torvalds}, commence à développer un \RefGlossaire{Système}{Kernel}{Noyau}{noyau} et demande aux personnes intéressées d'y contribuer. La licence GPL a été publiée à la même époque et \Personne{Linus}{Torvalds} s'est laissé persuader\NotePage{Ceci est une longue histoire\ldots{}} de placer son noyau sous cette dernière. Le \RefGlossaire{Système}{OS}{Système d'exploitation}{système d'exploitation} actuellement connu est donc un assemblage des outils GNU fonctionnant sur un \RefGlossaire{Système}{Kernel}{Noyau}{noyau} Linux, on parle donc de GNU/Linux avec le slash, «~/~» pour «~GNU sur Linux~».\par
GNU/Linux est un \RefGlossaire{Système}{OS}{Système d'exploitation}{système d'exploitation} complètement \AddIndex{Philosophie}{-}{Libre}{Libre} et performant. Il est hautement configurable. Il ne dépend pas d'une multinationale. Il est supporté par une grande communauté d'utilisateurs souvent prêts à vous aider. Quel que soit votre domaine de compétence, vous pouvez participer à l'amélioration de GNU/Linux pour que ce dernier évolue dans votre intérêt. Il n'y a pas de \RefGlossaire{Problèmes légaux}{GDN}{DRM}{DRM}\NotePage{Mécanisme de contrôle} cachés dans GNU/Linux. Ce n'est pas un simple logiciel gratuit, mais un \RefGlossaire{Philosophie}{Libre}{Logiciel Libre}{Logiciel Libre}, ce qui garantit qu'il restera accessible gratuitement pour tous, sans discrimination. De plus, la mascotte de Linux est un manchot\NotePage{Et non un pingouin car pinguin = manchot en Français, je le note pour dit !} du nom de Tux, et ça, c'est vraiment cool ;-) !\par
Beaucoup d'arguments pourraient encore être listés ici. Mais le plus important réside dans le fait de lui laisser sa chance, en lui offrant quelques heures de votre temps. On ne sait jamais, il pourrait bien vous offrir en retour une expérience intéressante, pour ne pas dire hors du commun.\par
\section{Qu'est-ce qu'une distribution ?}
En réalité, si on vous livrait le \RefGlossaire{Système}{Kernel}{Noyau}{noyau} Linux seul, accompagné des outils GNU de base, vous seriez bien avancé : pas d'interface graphique, juste quelques commandes, bref, votre \RefGlossaire{Système}{OS}{Système d'exploitation}{système d'exploitation} serait inexploitable, un comble, non ? C'est pour cela qu'existent des distributions Linux qui contiennent le \RefGlossaire{Système}{Kernel}{Noyau}{noyau} Linux, les outils GNU, plus un ensemble de logiciels qu'elles ont choisi de supporter. Ceux-ci sont testés et compilés pour vous\NotePage{Pour plus d'informations sur la différence entre code source, code binaire et la compilation, veuillez vous référer à la section \ref{RefSourceBinaire}}. La plupart des distributions contiennent un système d'installation de logiciel simplifié qui leur est -- malheureusement -- propre. Vous avez déjà dû voir qu'il existe de très nombreuses distributions\NotePage{Une liste complète et un «~classement d'utilisateurs~» des distributions sont disponibles sur \url{http://distrowatch.com}.}: Mandriva, Red Hat Fedora, Debian, Gentoo, OpenSuse et\ldots{} Ubuntu !\par
Alors pourquoi autant de distributions, me direz-vous ? En fait, chaque distribution a sa cible : certaines sont orientées sur la facilité d'utilisation, d'autres sont pour les véritables «~\RefGlossaire{Jargon}{-}{Geek}{geeks}~», certaines sont spécialisées pour l'utilisation dans le domaine scolaire ou musical\NotePage{Orientation MAO : Musique Assistée par Ordinateur}, d'autres ont des fonctionnalités optimisées pour les netbooks et leur écran réduit, d'autres encore se veulent très légères et fonctionnent sur des PC antédiluviens ou encore optimisées sur des écrans de tailles réduites\ldots{} Vous voyez qu'il peut y avoir -- presque ! -- autant de distributions que de cas d'utilisation !\par
\section{Pourquoi la distribution Ubuntu en particulier ?}
Quelques raisons parmi tant d'autres :
\begin{itemize}
\item Son rapprochement avec le projet \AddIndex{Environnement graphique}{-}{GNOME}{GNOME} qui propose une interface simple et intuitive. Pour ceux qui ne le sauraient pas, GNU/Linux vous permet de choisir votre environnement graphique\NotePage{Nous verrons cette notion un peu plus tard}.
\item Sa parenté avec le projet Debian, distribution reconnue pour sa très grande stabilité, excellente mais pouvant sembler relativement difficile d'accès. On peut voir Ubuntu comme une distribution rendant Debian accessible au grand public\NotePage{Pitié, que les debianistes ne me jettent pas la pierre !}.
\item Sa communauté très active. Une question posée sur le forum ne reste pas longtemps sans réponse(s). La documentation française est très fournie et librement accessible.
\item Sa fréquence de mise à jour fixe\NotePage{Tous les 6 mois, contrairement à Debian\ldots{} pour une version stable :-)}. On sait à quoi s'attendre. Si un logiciel n'est pas intégré dans sa dernière version vous savez combien de temps attendre pour l'obtenir dans la suivante. De plus, la mécanique de gestion des logiciels héritée de Debian vous permet d'installer d'autres logiciels tiers et/ou plus récents, très simplement.
\item Pas de compte root\NotePage{Compte administrateur} : l'utilisateur qui installe la distribution est considéré comme un utilisateur spécial qui peut hériter des droits d'administrateur via une commande particulière\NotePage{Puisque je sens chez vous une irrésistible soif de connaissance, je vous la donne tout de suite : \Commande{sudo}}. Ainsi, en utilisation courante, les programmes que l'on exécute ne peuvent pas altérer la bonne configuration du système. Ceci augmente considérablement la sécurité du système.
\item Ubuntu est gratuit et simple à installer.
\item \Personne{Mark}{Shuttleworth}, fondateur d'Ubuntu, l'indique lui-même : «~Chaque manipulation réalisée à l'aide de lignes de commande est un bug qu'il faut corriger~». Cela montre la forte orientation vers l'utilisateur de cette distribution.
\item Le site francophone de la communauté Ubuntu rassemble une communauté vraiment active --- actuellement \NbMembresUbuntufr{} membres. Vérifiez-le par vous même sur \url{http://www.ubuntu-fr.org}.
\end{itemize}
\section{Courte présentation d'Ubuntu}
\ImgPar{r}{3.34}{corps/chapitre1/img/Ubuntu_Logo.png}
Cette distribution a été fondée par un milliardaire sud-africain : \Personne{Mark}{Shuttleworth}. Développeur Debian au milieu des années 1990, il a été fortement médiatisé pour avoir été le deuxième milliardaire\NotePage{Mais premier Africain !} à voyager dans l'espace. Il créa Ubuntu en 2004 dont l'objectif avoué est de populariser Linux via sa société Canonical Ltd. Ensuite, il fonda la Ubuntu Foundation dès 2005 en lui apportant une contribution initiale de 10 millions de dollars afin de rémunérer les développeurs d'Ubuntu. Aujourd'hui, \Personne{Mark}{Shuttleworth} a donné plus de la moitié de sa fortune à des œuvres de charité.\par
«~Ubuntu~» est un ancien mot africain qui signifie «~humanité aux autres~». Ubuntu signifie également «~Je suis ce que je suis grâce à ce que nous sommes tous~». La distribution Ubuntu Linux apporte l'esprit Ubuntu au monde logiciel.\par
Ubuntu est parti de ce constat qui constitue le fameux bug numéro 1\NotePage{Bug \#1} du Launchpad\NotePage{Site sur lequel on peut rapporter un bug sur une application} d'Ubuntu  : \url{https://launchpad.net/distros/ubuntu/+bug/1}.\\
En voici une traduction maladroite, j'en conviens, réalisée par mes soins :
\begin{citationlongue}{\Personne{Mark}{Shuttleworth}, le 20 août 2004}
Microsoft détient une large majorité sur le marché des ordinateurs de bureau. Ceci constitue un bug, et Ubuntu est là pour le réparer.\\
Microsoft détient une large majorité sur le marché. Le \RefGlossaire{Philosophie}{-}{Logiciel Propriétaire}{logiciel propriétaire} freine l'innovation dans l'industrie informatique, ce qui a pour effet de restreindre l'accès à l'informatique à une petite part de la population mondiale et de limiter la capacité des développeurs à atteindre leur plein potentiel. Ce bug est très évident, notamment dans l'industrie du PC.
Voici la démarche à suivre pour reproduire le bug :
\begin{enumerate}
\item Visitez un magasin d'informatique
\item Observez que la majorité des PC à vendre ont des \RefGlossaire{Philosophie}{-}{Logiciel Propriétaire}{logiciels propriétaires} pré-installés.
\item Remarquez que très peu de PC sont vendus avec Ubuntu et/ou des \RefGlossaire{Philosophie}{Libre}{Logiciel Libre}{Logiciels Libres} pré-installés.\\
Ce qui devrait arriver prochainement :
\item La majorité des ordinateurs à vendre devraient inclure seulement les \RefGlossaire{Philosophie}{Libre}{Logiciel Libre}{Logiciels Libres} comme Ubuntu.
\item Ubuntu devrait faire l'objet d'une médiatisation de manière à ce que ses capacités étonnantes et ses bienfaits soient visibles et connus de tous.
\item Le système devrait, au fur et à mesure, devenir de plus en plus tourné vers l'utilisateur.
\end{enumerate}
\end{citationlongue}
Ce bug est connu, confirmé, placé au niveau d'importance critique et assigné à \Personne{Mark}{Shuttleworth} :-).\par
\begin{nota}
Le point 2 est «~normalement~» interdit en France si on ne propose pas comme alternative le même matériel sans logiciels pré-installés. Contrairement à ce que la plupart des gens pensent, ces logiciels ne sont pas gratuits et coûtent environ le tiers du prix global. Cela s'appelle de la vente liée car on subordonne la prestation d'un service -- une licence de logiciel -- à l'achat d'un produit -- l'ordinateur dans ce cas, mais l'administration française ne semble pas vouloir faire bouger ce dossier. Pour plus de renseignements sur ce sujet, visitez le site \url{http://www.racketiciel.info/}.
\end{nota}
\section{Les versions d'Ubuntu}
\label{RefVersionUbuntu}
\subsection{Nom et numéro de version}
La numérotation des versions de Ubuntu est basée sur l'année et le mois de sa sortie [A.MM]. La première version de Ubuntu, sortie en octobre 2004, portait le numéro de version 4.10. La version suivante, sortie en avril 2005, portait le numéro 5.04 et ainsi de suite. La première version dite LTS\NotePage{Long Term Support : Support à long terme}, 6.06, était sortie en juin 2006 et la version actuelle, 10.04, date donc d'avril 2010. On lui associe souvent un nom de code, formé d'un nom d'animal précédé d'un adjectif, tous deux commençant par la même lettre. La première version était la Warty Warthog\NotePage{Le Hérisson Verruqueux}.
%La dernière version LTS, Hardy Heron\NotePage{Le Héron Hardi} est sortie en avril 2008, la version actuelle a comme nom de code Karmic Koala\NotePage{Alias le Koala avec du Karma.}.
La version actuelle a comme nom de code Lucid Lynx\NotePage{Alias le lynx lucide} et est une LTS.
Chaque version de Ubuntu a une combinaison unique de ses composantes -- le \RefGlossaire{Système}{Kernel}{Noyau}{noyau}, le \AddIndex{Système}{-}{Serveur graphique}{serveur graphique} X11, l'environnement de bureau \AddIndex{Environnement graphique}{-}{GNOME}{GNOME}, GCC, libc\ldots{} -- qui ont toutes des numéros de version différents et n'ayant pas tous la même signification. Baser le chiffre de la version sur les composantes du système aurait eu peu de sens. Ubuntu préfère plutôt donner une idée de la date à laquelle la version a été stabilisée, mise en production.
\subsection{Mises à jour}
Contrairement à d'autres distributions Linux, lorsqu'une version de Ubuntu est stabilisée, les versions des logiciels qu'elle inclut sont gelées. Ainsi, si une nouvelle version stable d'un logiciel ou d'une bibliothèque quelconque est disponible après la sortie définitive d'une version de Ubuntu, l'intégration de celle-ci à Ubuntu se produira dans la prochaine mouture de l'\RefGlossaire{Système}{Système d'exploitation}{OS}{OS}.\par
Cette manière de procéder assure une meilleure homogénéité des versions pour du support technique de la part de Canonical Ltd. et de ses partenaires. Cette caractéristique est certainement requise pour un déploiement de Ubuntu en entreprise. De plus, elle assure que le système, dans sa version actuelle, reste stable et fonctionnel.\par
Les seules mises à jour publiées pour les versions stables encore supportées sont des mises à jour de sécurité, corrigeant bogues, failles, et autres problèmes de fonctionnement de l'actuelle version, éventuellement prises des nouvelles versions, mais adaptées aux versions  plus anciennes.
\subsection{Fréquence des sorties et durée de vie}
Des versions stables de Ubuntu sortent deux fois par an, aux mois d'avril et d'octobre. Le développement de Ubuntu est lié au développement de l'environnement de bureau \AddIndex{Environnement graphique}{-}{GNOME}{GNOME} : la version finale de Ubuntu sort environ un mois après la publication d'une nouvelle version stable de \AddIndex{Environnement graphique}{-}{GNOME}{GNOME}. Ubuntu suit donc un cycle de développement de six mois.\par
Tous les 2 ans sort une version LTS pour laquelle des mises à jour de sécurité, des correctifs et du support technique seront publiés pendant 3 ans en ce qui concerne une utilisation de type poste de travail ou de 5 ans pour une utilisation de type serveur. La première version à avoir bénéficié de ce support est la version Ubuntu 6.06 «~The Dapper Drake~», la dernière est la version actuelle 10.04, «~Lucid Lynx~».
\subsection{Je ne veux pas renoncer à mon Windows !}
\label{RefKeepWindows}
Vous ne voulez pas vous séparer complètement de Microsoft Windows ? Vraiment ? GNU/Linux n'est pas un sauvage\NotePage{Lui ;-)} : il tolère très bien la colocation. C'est-à-dire que vous pouvez très bien avoir, sur le même ordinateur, une -- ou plusieurs -- partition(s) Linux et une -- ou plusieurs -- partition(s) Windows. Sachez tout d'abord qu'une partition n'a rien à voir avec de la musique, bien que vous soyez le chef d'orchestre de votre ordinateur ! En effet, il s'agit d'une zone mémoire découpée sur un disque dur, donc une portion de ce dernier. On peut diviser son disque dur en plusieurs partitions, et lorsque l'on écrit une donnée sur une portion du disque dur, cela n'affecte en rien ce qui existe sur les autres partitions. Vous pouvez donc installer sans aucune crainte une distribution GNU/Linux et garder votre «~précieux~» Microsoft Windows.\par
Lorsque vous allumerez votre ordinateur, un écran vous permettra de choisir quel \RefGlossaire{Système}{OS}{Système d'exploitation}{système d'exploitation} vous souhaitez utiliser. Cet écran de connexion est généré par un logiciel appelé GRUB qui s'installe dans le \RefGlossaire{Système}{Boot loader}{Secteur d'amorce}{secteur d'amorce} de votre disque dur principal. Vous trouverez un aperçu\NotePage{Il est possible aussi de rajouter des couleurs, voire une photo en fond d'écran !} de ce que vous obtiendrez alors en allumant votre ordinateur par l'image \ref{ImgGrub}.\par
\ImgCentree{12}{corps/chapitre1/img/Grub.png}{Vous pouvez choisir ici quel système démarrer}{ImgGrub}
Pour obtenir cela, vous devez :
\begin{itemize}
\item Faire un peu de place sur votre disque dur,
\item Sauvegarder vos données sensibles\NotePage{Comme vos photos personnelles, documents importants\ldots{}}; cette étape n'est pas obligatoire mais vivement conseillée,
\item Défragmenter vos partitions Windows,
\item Repartitionner votre disque dur\NotePage{Couper en portions votre disque dur} pour dégager un espace libre où installer Linux. Pour cette étape je vous conseille Gparted-Live\NotePage{À la place de Partition Magic puisqu'il est gratuit, se télécharge vite -- 31Mio -- et ne nécessite pas d'installation} si vous voulez l'effectuer avant l'installation de Linux. Sinon, pas de panique, l'installateur d'Ubuntu inclut cette étape.
\end{itemize}
Vous pourrez ainsi profiter sereinement de GNU/Linux sans peur de casser votre Windows.
\section{Mes logiciels, mes jeux, mon matériel\ldots{}}
\subsection{Les logiciels}
Si vous utilisez Firefox, Thunderbird, The GIMP,\ldots{} sachez que ces programmes existent sous Linux. Il s'agit même de leur \RefGlossaire{Système}{Système d'exploitation}{OS}{OS} natif\NotePage{Par conséquent, ils tournent souvent plus rapidement} ! Si vous utilisez Photoshop, Outlook, Moviemaker, Nero Burning Rom, certains peuvent tourner sous GNU/Linux mais ce n'est pas forcément très simple à mettre en place. Enfin, il existe presque toujours des logiciels équivalents\NotePage{Plus ou moins différents mais remplissant des tâches identiques}, voire même supérieurs.
\subsection{Les jeux commerciaux}
Ils sont rarement compatibles avec GNU/Linux bien que Cedega ou encore Wine permettent d'en faire fonctionner certains\NotePage{Pour plus d'informations sur ce sujet, référez-vous à la section \ref{RefWineJeux}.}. Toutefois, les sorties d'une version de UT2004, Neverwinternight, Quake 3 et 4 pour GNU/Linux sont de bon augure pour la suite\ldots{}\par
\subsection{Les jeux libres}
Il existe également de nombreux jeux \AddIndex{Philosophie}{-}{Libre}{Libres} de très bonne qualité comme vous pourrez le voir au chapitre \ref{RefChapJeux}. Quelques images de la figure \ref{ImgPresentationJeuxLinux} sont là pour vous mettre l'eau à la bouche.
\PresentationJeux
\subsection{Votre matériel sera-t-il reconnu ?}
Certains périphériques n'ont pas de \RefGlossaire{Système}{Pilote}{Driver}{drivers} écrits pour GNU/Linux, du fait de certains constructeurs de matériel, qui n'en fournissent -- pour des questions de coût -- que pour le \RefGlossaire{Système}{OS}{Système d'exploitation}{système d'exploitation} Windows. Les développeurs GNU/Linux sont donc obligés de créer eux-mêmes le \RefGlossaire{Système}{Pilote}{Driver}{driver} pour leur \RefGlossaire{Système}{OS}{Système d'exploitation}{système d'exploitation}. Bien évidemment, moins une firme fournit de documentation sur son matériel, plus la tâche est ardue\NotePage{Les programmeurs sont obligés de «~tâtonner~»} et à la vue du nombre de matériels existants, vous pouvez imaginer l'ampleur du problème !\par
Néanmoins, ne vous inquiétez pas car l'absence d'un \RefGlossaire{Système}{Pilote}{Driver}{driver} pour votre matériel sous GNU/Linux concerne principalement le matériel exotique ou très récent. Dans ce dernier cas, laissez juste le temps aux développeurs\NotePage{Souvent bénévoles, je le rappelle} de réussir à écrire un \RefGlossaire{Système}{Pilote}{Driver}{driver}. Apporter votre aide, en faisant des tests par exemple, ne peut que faire accélérer le processus. Une fois que tous les constructeurs auront compris que GNU/Linux prend de plus en plus d'importance -- et on y vient -- peut-être aura-t-on une majorité de matériel «~GNU/Linux compatible~». D'ici là, je ne peux que vous recommander, avant l'achat d'un nouveau matériel, de vérifier\NotePage{Sur l'\RefGlossaire{-}{L'Internet}{Internet}{Internet} par exemple} qu'il est «~GNU/Linux compatible~» et de favoriser les maisons fournissant des \RefGlossaire{Système}{Pilote}{Driver}{drivers} pour GNU/Linux.\par
De plus, Ubuntu s'installe à partir d'un «~LiveCD~» appelé «~Desktop CD~». Autrement dit, le CD-ROM lance un système Ubuntu complet avant même de lancer l'installation. Ainsi, si vous avez réussi à lancer le système, c'est que les principaux composants de votre machine fonctionnent avec GNU/Linux.
\section{Quelle est la relation entre Ubuntu et Canonical ?}
Canonical, par l'entremise de \Personne{Mark}{Shuttleworth} est le plus gros sponsor d'Ubuntu, dans le sens où la plupart des développeurs de la distribution sont employés à plein temps par cette société et qu'elle organise et sponsorise les UDS (Ubuntu Developer Submit), tous les 6 mois, où des discussions autour du futur de la distribution ont lieu. Très logiquement, elle en assure un service de support technique (payant, donc plus dédié aux entreprises) et la certification.\par
Canonical travaille également sur d'autres projets comme Bazaar (\url{http://bazaar-vcs.org/}) ainsi que Launchpad (\url{https://launchpad.net/}).\par
Enfin, la société a mis en place un programme de partenariat avec des entreprises commerciales, qui s'établit sur plusieurs pistes et degrés progressifs. Plus le partenaire fournit de services liés à Ubuntu, et plus son niveau relationnel avec Canonical devient important, donnant accès à de plus en plus de bénéfices. Ce programme de partenariat concerne exclusivement Ubuntu parmi les projets Canonical (Ubuntu restant le projet phare de la société).
