\chapitre{Découverte et installation}{Maintenant que nous sommes}{ un peu plus renseignés sur les fondements du \RefGlossaire{Système}{OS}{Système d'exploitation}{système d'exploitation} GNU/Linux, et plus particulièrement sur la distribution Ubuntu avec ses atouts et ses rares inconvénients, vous vous demandez sûrement comment l'obtenir et l'installer afin de l'essayer sans plus tarder ? Dans ce chapitre, nous vous indiquerons précisément où vous procurer Ubuntu, la tester sans rien toucher à vos disques durs, et ensuite, comment procéder à l'installation. Nous aborderons également un point essentiel : la cohabitation avec votre \RefGlossaire{Système}{OS}{Système d'exploitation}{système d'exploitation} Windows, et le partage de données entre ces systèmes.}
\section{Le CD}
\subsection{Obtenir le CD de Ubuntu}
Le CD, appelé «~Desktop CD~», permet non seulement l'installation d'Ubuntu mais aussi d'ouvrir une session «~Live~». Une session Live permet de tester une distribution Linux sans installer quoi que ce soit sur le disque dur. Le système est plus lent qu'une distribution Linux installée, mais au moins, vous ne touchez pas à votre système. Il existe plusieurs méthodes pour l'obtenir :
\subsubsection{Pour les patients qui n'ont pas de connexion ADSL}
Commandez votre CD sur le site «~Shipit~» d'Ubuntu à cette adresse : \url{http://shipit.ubuntu.com}. En procédant ainsi, vous recevrez un colis dans les 3 à 4 semaines qui suivent votre demande. Le Shipit est gratuit et vous permet de commander la version LTS de Ubuntu Lucid Lynx (10.04) et
 la dernière version en date, Maverick Meerkat (10.10).\par
Cependant, si vous souhaitez soutenir l'association ubuntu-fr, cette dernière vous permet également d'obtenir un CD complètement francisé dès l'installation à un prix modique. Pour cela (et voir plus de goodies), rendez-vous sur \url{http://enventelibre.org/}.\par
Vous pouvez également venir à l'une de nos Ubuntu Party. En plus de bénéficier d'une ambiance très conviviale, vous pourrez discuter avec de véritables passionnés du Logiciel \AddIndex{Philosophie}{-}{Libre}{Libre}. Pour savoir où et quand se déroule cette fête près de chez vous, retrouvez-nous sur \url{http://www.ubuntu-party.org}.
\subsubsection{Pour les impatients qui n'ont pas de connexion ADSL}
Rendez-vous dans le G.U.L.\NotePage{Groupe d'Utilisateurs Linux} le plus proche de chez vous et demandez s'ils n'ont pas un CD de Ubuntu 10.10 Maverick Meerkat sous la main. Consultez la liste des G.U.L. francophones sur \url{http://www.aful.org/gul/liste} pour savoir où trouver ces associations. Ubuntu-fr, l'association francophone des utilisateurs d'Ubuntu en propose également en version intégralement francisée d'office -- plus d'information sur \url{http://www.ubuntu-fr.org}
\subsubsection{Pour ceux qui ont une connexion ADSL}
\begin{itemize}
\item Rendez-vous sur le site officiel, dans la section download à cette adresse : \url{http://www.ubuntu.com/getubuntu/download}.
\item Choisissez la version désirée --- traditionnellement le «~Desktop CD~» de la dernière version% sauf si vous cherchez la stabilité sans faille et dans ce cas, vous prendrez la version marquée LTS\NotePage{Voir la remarque sur les différentes versions dans la section \ref{RefVersionUbuntu}}.
\item Sélectionnez ensuite l'architecture correspondante à votre ordinateur. Cette architecture correspond la plupart du temps -- et c'est sûrement le cas si vous ne connaissez pas cette notion -- à celle indiquée sous le nom «~Standard personal~» (architecture i386, pour les processeurs 32 bits), «~64bit AMD and Intel computers~» (pour les processeurs 64 bits) ou encore ARM.
\item Choisissez un des serveurs de téléchargement proposés. Le plus proche de chez vous fait généralement l'affaire.
\item Si vous savez en quoi cela consiste, vérifiez  que le fichier téléchargé n’est pas corrompu en comparant l’empreinte md5sum\NotePage{Une petite recherche Google vous aidera, le principe est très simple et vous évite de graver un cd inutilisable}. En cas de doute,  re-téléchargez-le.
\item Gravez le fichier contenant l'extension .iso. Si vous avez le logiciel WinRAR\NotePage{Ou équivalent comme 7zip qui fait le même boulot «~librement~»}, le fichier .iso aura une \RefGlossaire{-}{-}{Icône}{icône} correspondant à ce logiciel car ce dernier peut extraire les fichiers de ce type d'archive, surtout n'utilisez pas cette possibilité pour faire un disque de données, votre CD ne serait pas bootable, c'est-à-dire qu'il ne pourra pas démarrer lorsque vous mettrez votre ordinateur sous tension.
\end{itemize}
\begin{nota}
Attention, il faut utiliser la fonctionnalité correspondante de votre logiciel de gravure (le plus souvent nommée «~Graver une image~» ou «~Graver à partir d'une image disque~»). Il ne faut en aucun cas extraire les fichiers de l'archive ou graver directement le fichier iso dans un CD de données.
\end{nota}
\begin{nota}
En parlant uniquement du «~Desktop CD~», en fait, je vous ai quelque peu menti par omission\NotePage{Ah bah\dots{} ça commence bien !}. Il existe en effet l'Alternate CD permettant une installation directe. Vous pourriez avoir besoin de ce CD dans principalement deux cas :\par
\begin{itemize}
\item Le «~Desktop CD~» ne démarre pas sur votre PC ou votre PC possède moins de 384 Mio\NotePage{Et non Mo, pour plus de renseignements sur ce point, voir \ref{RefMoMio}.} de \RefGlossaire{-}{Mémoire vive}{RAM}{RAM}. Dans ce cas, l'installation directe, toujours avec le même CD, peut s'imposer en sélectionnant au démarrage «~Installer Ubuntu~» et en abandonnant la section «~Session Live~» du point \ref{RefSessionLive}. Ubuntu demande un minimum de 256 Mio pour tourner correctement, une fois installé.
\item Vous êtes un utilisateur avancé\NotePage{Pourquoi lisez-vous ce livre alors? :-)} et vous préférez avoir quelques options supplémentaires de configuration que ne propose pas l'installateur graphique\NotePage{Très simplifié, certes} du Live CD.
\end{itemize}
\end{nota}
Vous pouvez également trouver ce CD sur le site officiel d'Ubuntu, au sein de la section download\NotePage{\url{http://www.ubuntu.com/getubuntu/download}} en cochant la case -- tout en bas -- «~Check here if you need the alternate desktop CD. This CD does not include the Live CD, instead it uses a text-based installer.~».\par
Dans la suite de cette documentation, nous détaillerons l'installation pour le Desktop CD. Sachez cependant que l'installation par l'Alternate CD est très similaire, même si graphiquement, moins jolie.
\subsection{J'ai mon CD, que faire maintenant ?}
\subsubsection{Insérez le CD d'installation}
Deux cas :
\begin{itemize}
\item Votre ordinateur est allumé, c'est facile : vous ouvrez le lecteur, insérez le CD, refermez et redémarrez en espérant que votre ordinateur est configuré pour démarrer à partir de votre lecteur CD-ROM (voir la section suivante).
\item Votre ordinateur est éteint. Là, c'est relativement amusant car vous ne pouvez théoriquement pas ouvrir votre lecteur de CD et pourtant le CD doit être en place au démarrage. La solution la plus simple est d'entrer dans le \RefGlossaire{Système}{-}{BIOS}{BIOS} pour avoir le temps de mettre le CD-ROM dans le lecteur avant que la procédure de démarrage ne se poursuive. Vous gagnerez ainsi du temps ;). Au démarrage de l'ordinateur, une ligne doit vous indiquer comment activer le «~setup~». Généralement il s'agit d'une des touches : \Touche{Suppr}\NotePage{\Touche{Del} en anglais}, \Touche{F2}, \Touche{Échap} ou encore \Touche{F8}.
\end{itemize}
\begin{nota}
Pour les bricoleurs du dimanche (ou de tout autre jour de la semaine, d'ailleurs) : il est en réalité possible d'ouvrir votre lecteur de CD mécaniquement, même à l'arrêt, en introduisant délicatement un trombone déplié dans le petit orifice en façade du boitier, généralement à droite du plateau coulissant et le déverrouiller en tâtonnant (c’est la procédure utilisée à l’arrêt par tous les ateliers). 
\end{nota}
\ImgCentree{4.45}{corps/chapitre2/img/ChoisirLecteur.png}{Vous pouvez choisir ici de démarrer sur le CD-ROM}{ImgChooseStart}
\begin{nota}
Une option vous permet aussi d'installer Ubuntu sous Windows : plus simple me direz-vous et cependant je vous le déconseille. En effet, Ubuntu fonctionnera alors en tant que  «~surcouche~» de Windows et consommera par conséquent énormément de \RefGlossaire{-}{RAM}{Mémoire vive}{mémoire vive}, sera plus lent et certaines incompatibilités matérielles peuvent apparaître. Bref, rien ne vaut une bonne installation indépendante comme décrite dans la suite de ce chapitre.
\end{nota}
\subsubsection{Démarrez sur le CD}
La \RefGlossaire{Système}{-}{Séquence de démarrage}{séquence de démarrage} est l'ordre dans lequel votre PC va consulter les périphériques à la recherche d'un \RefGlossaire{Système}{OS}{Système d'exploitation}{système d'exploitation}. Pour que l'installation commence, votre ordinateur doit activer le lecteur CD avant le disque dur. Si votre ordinateur n'affiche pas le logo Ubuntu après un redémarrage, alors que le CD-ROM se trouvait à l'intérieur de votre lecteur, rendez-vous dans le  \RefGlossaire{Système}{-}{BIOS}{BIOS} pour modifier cette séquence.\par
\subsubsubsection{Modifier la séquence de démarrage}
Sur les ordinateurs récents, il est souvent possible d'appuyer simplement sur la touche \Touche{F12} au démarrage pour pouvoir choisir le lecteur à activer en premier comme le montre la figure \ref{ImgChooseStart}. Sélectionnez CD-ROM/DVD Drive (ou toute ligne à l'orthographe proche), validez et vous vous ouvrirez les portes de l'univers Ubuntu. La figure \ref{ImgLiveCD1} est une copie de l'écran auquel vous devriez faire face à présent.\par
Si cela ne fonctionne pas ou que vous ne vous sentez pas à même de configurer \RefGlossaire{Système}{-}{Séquence de démarrage}{l'ordre de démarrage}, Ubuntu peut effectuer cette tâche pour vous. Pour cela, insérez votre CD-ROM lorsque Windows est démarré. La fenêtre \ref{ImgHelpBootOnCD} apparaîtra alors et vous proposera plusieurs options en anglais. Choississez «~Démo and full installation~». Redémarrer la machine sera alors nécessaire et la troisième option «~Help me to boot from CD~» devra alors être sélectionnée.
\ImgCentree{8}{corps/chapitre2/img/HelpBootSurCD.png}{Vous pouvez choisir ici de démarrer sur le CD-ROM}{ImgHelpBootOnCD}
\subsubsubsection{Premier écran et lancement d'Ubuntu}
Un premier écran \ref{ImgLiveCD0} s'affiche alors pendant quelques secondes. Si vous n'appuyez sur aucune touche, Ubuntu démarrera (voir \ref{ImgLiveCD3}), et il vous sera présenté un écran \ref{ImgLiveCD4-2} vous permettant de choisir votre langue, puis soit d'installer, soit d'essayer d'Ubuntu sans effectuer immédiatement l'installation. Si l'installation est sélectionnée, vous pouvez passer à la section \ref{RefInstall}, sinon, vous pouvez passer à \ref{RefSessionLive} (choix conseillé si c'est la première fois que vous désirez essayer ubuntu).\par
\ChoixDebutLive
Pour les utilisateurs plus avancés, si vous appuyez sur une touche lors du premier écran, un menu est automatiquement présenté sur lequel vous pourrez choisir la langue désirée par les touches fléchées puis en appuyant sur \Touche{Entrée}. Il vous est possible de revenir sur votre choix en appuyant sur la touche \Touche{F2}.\par
Vous obtiendrez ainsi un menu et une interface en Français (ou autres) comme montré sur la figure \ref{ImgLiveCD2}.\par
D'autres options sont également disponibles comme la désactivation de la gestion avancée d'énergie -- ACPI=off -- si jamais vous voyez que vous ne pouvez pas démarrer normalement Ubuntu.
\ChoixDebutInstall
\section{Session Live}
\label{RefSessionLive}
C'est parti, vous avez validé essayer Ubuntu par l'une des deux méthodes précédentes, une session live se lance. Soyez patient, la détection du matériel de votre ordinateur est en cours et une jolie icône de chargement est là pour vous faire patienter --- cf \ref{ImgLiveCD3}\ldots{}\par
\begin{nota}
Vous aurez noté l'option «~Installer Ubuntu~» sur les images \ref{ImgLiveCD2}. Cette option correspond à un raccourci vous amenant directement à installer votre nouveau système d'exploitation sans essai au préalable. Le résultat est le même que la démarche plus complète que je vous donne -- hormis que vous ne pourrez pas configurer votre connexion à l'\RefGlossaire{-}{L'Internet}{Internet}{Internet} par ce moyen-ci.\\
Les autres options permettent de vérifier la bonne intégration de votre CD-ROM, tester la mémoire vive de votre ordinateur ou encore de démarrer classiquement à partir du disque dur.
\end{nota}
Une fois l'interface graphique chargée, vous devriez trouver l'écran comme présenté en image \ref{ImgLiveCD4}.\par
\ImgCentree{12}{corps/chapitre2/img/LiveCD4.png}{L'interface graphique par défaut d'Ubuntu}{ImgLiveCD4}
Profitez-en pour découvrir votre futur \RefGlossaire{Système}{OS}{Système d'exploitation}{système d'exploitation}. C'est plus lent que si vous utilisiez une version installée\NotePage{Vous n'utilisez que votre lecteur de CD-ROM et la mémoire vive de l'ordinateur} mais grâce à cela vous pouvez tester un système Ubuntu Linux et constater que tout fonctionne avant d'installer et de modifier quoi que ce soit dans votre ordinateur.
\begin{nota}
Je vous conseille, si vous avez accès à l'\RefGlossaire{-}{L'Internet}{Internet}{Internet}, de configurer ce dernier si cela n'est pas déjà fait automatiquement. En effet, vous aurez ainsi les dernières mises à jour si vous décidez d'installer Ubuntu sur votre ordinateur, et ce, dès l'installation ! Essayez de lancer Firefox -- la première \RefGlossaire{-}{-}{Icône}{icône} à côté du menu «~Système~» -- pour le vérifier. Cela devrait être le cas, si vous êtes connecté par un câble Ethernet,  avec ou sans\NotePage{Oui, vous avez bien lu, même «~sans~», grâce à \RefGlossaire{Réseau}{Connexion réseau}{Avahi}{Avahi} que l'on peut choisir comme indiqué en \ref{RefAvahi}.} attribution des adresses par \RefGlossaire{Réseau}{Connexion réseau}{DHCP}{DHCP}. Dans tous les cas, veuillez vous référer à la section d'installation du réseau en \ref{RefInstallReseau}.
\end{nota}
\section{L'installation}
\label{RefInstall}
Un Live c'est beau, une installation c'est mieux ! Vous aimez Ubuntu et voulez l'installer ? Double-cliquez sur l'\RefGlossaire{-}{-}{Icône}{icône} «~Installer Ubuntu 10.10~» présente sur le bureau. L'installeur répondant au doux nom «~d'Ubiquity~» se lance\ldots{}
\begin{attention}
Petite remarque préalable sur les PCs tatoués.\label{RefTatouage}
Le tatouage est un dispositif, imposé par Microsoft, interdisant à la version Windows livrée avec votre ordinateur d'être installée sur un autre ordinateur. Concrètement il s'agit de quelques octets placés sur la carte mère, sur le disque dur -- en particulier le \RefGlossaire{Système}{Zone d'amorçage}{MBR}{MBR}-- sur la partition de restauration Windows, sur les CD de réinstallation, voire aussi dans le \RefGlossaire{Système}{-}{BIOS}{BIOS}, et qui communiquent entre eux. Si l'un de ces éléments est changé, votre Windows ne pourra plus être réinstallé et au pire, plus démarrer non plus. Ce dispositif est utilisé par de grandes marques de constructeurs d'ordinateurs. Cela est d'autant plus vrai depuis l'arrivée de Windows VISTA.\\
L'installation d'Ubuntu depuis le «~Desktop CD~» en dualboot avec Windows installe GRUB\NotePage{Le logiciel vous permettant de choisir quel \RefGlossaire{Système}{OS}{Système d'exploitation}{système d'exploitation} démarrer} sur le \RefGlossaire{Système}{Zone d'amorçage}{MBR}{MBR} du disque dur amorçable. Ce faisant, GRUB\NotePage{GRand Unified \RefGlossaire{Système}{Chargeur d'amorce}{Boot loader}{Boot loader}} écrase le \RefGlossaire{Système}{Zone d'amorçage}{MBR}{MBR}, vous perdez alors le tatouage, ce qui entraîne l'impossibilité de démarrer Windows. Si vous pensez que votre ordinateur est tatoué, des méthodes existent pour que vous ne perdiez pas votre Windows. Les procédures évoluant trop rapidement, elles ne peuvent actuellement être éditées dans un livre. Vous trouverez, ici, des liens vers la description de celles-ci : \url{http://doc.ubuntu-fr.org/cohabitation_ubuntu_windows}.
\end{attention}\par
Ce point étant clarifié, c'est parti pour une installation.
\subsection{Bienvenue}
Choisissez en premier lieu la langue d'installation comme montré en figure \ref{ImgInstall1}. Jusqu'ici, rien de bien compliqué. Quoique\ldots{}
\ImgCentree{8.78}{corps/chapitre2/img/Install1.png}{Choix de la langue}{ImgInstall1}
\subsection{Préparation de l'installation de Ubuntu}
L'écran suivant vérifie les points clefs pour une installation dans de bonne condition de Ubuntu, celles-ci étant :
\begin{itemize}
\item posséder assez de place sur le disque dur
\item dans le cas d'un ordinateur portable, être relié sur secteur
\item être connecté à Internet
\end{itemize}
Si vous avez accès à une connexion Internet, il vous est alors possible de sélectionner l'installation automatique de toutes les mises à jour déjà disponibles au terme de celle-ci, ainsi que d'installer le support pour les logiciels tiers comme le support MP3. Je pense que ces deux options répondent à la plupart des cas d'utilisation d'un nouvel utilisateur de Ubuntu. N'hésitez donc pas !
\ImgCentree{8.78}{corps/chapitre2/img/Install2.png}{Vérification des conditions d'installation}{ImgInstall2}
\subsection{Allouer de l'espace disque}
\label{RefFormatPartition}
\subsubsection{À l'attaque}
Cette suite d'écrans est la plus importante. Le premier écran vous propose plusieurs choix:
\begin{itemize}
\item La première option «~Installer à côté d'autres systèmes d'exploitation~», vous amènera à l'écran \ref{ImgInstall4}. Il vous affiche sous forme graphique les \RefGlossaire{Système}{OS}{Système d'exploitation}{systèmes d'exploitation} disponibles sur le système en y ajoutant Ubuntu. Vous pouvez alors sur ce dernier changer l'espace donné à chaque système d'exploitation sur votre système en fonction de la place disponible. Il est également possible de choisir un autre disque dur si votre ordinateur en comprend plusieurs. N'effectuez cette opération que si vous avez récemment défragmenté votre disque dur Windows. On vous laisse toujours la possibilité d'utiliser une partition ou un disque dur entier
\item Si vous souhaitez assigner un disque complet à GNU/Linux sur la machine, il vous suffit ici de sélectionner : «~Tout effacer et utiliser le disque entier~». Vous pourrez ensuite choisir quel disque dur utiliser.
\item Si vous avez préalablement dégagé un espace libre sur votre disque dur, vous pourrez sélectionner «~Utiliser le plus grand espace disque disponible~». (Cette option n'est éventuellement pas disponible si vous n'avez pas de place non formatée)
\item Si vous voulez controler finement votre installation, ce qui n'est pas conseillé pour les débutants, il va falloir en passer par le partitionnement manuel, étape que beaucoup redoutent. Mais n'ayez crainte, l'installeur Ubuntu propose une interface graphique qui simplifie grandement la chose comme nous allons le voir.
\end{itemize}\par
\ImgCentree{8.78}{corps/chapitre2/img/Install3.png}{Choix du mode de partitionnement}{ImgInstall3}
\ImgCentree{8.78}{corps/chapitre2/img/Install4.png}{Partitionnement automatisé}{ImgInstall4}
\subsubsection{Partionnement manuel}
Que faut-il faire pour partitionner correctement son disque ? Cela reste assez complexe à résumer puisque tout dépend de vos besoins et de votre matériel. Toutefois, voici quelques conseils :
\begin{itemize}
\item Si vous souhaitez faire cohabiter Windows et Linux, vous devrez dégager un espace libre en redimensionnant l'une des partitions de Windows comme indiqué en \ref{RefKeepWindows}.
\item Vous devez créer ensuite deux partitions ou plus :
\Partitionnement
\begin{itemize}
\item La partition principale symbolisée par «~\Chemin{/}\NotePage{Appelé «~slash~», root, ou encore racine du système}~», qui accueillera le système et qui doit être au minimum de 3 Gio\NotePage{Et non Go, une fois de plus, cf. \ref{RefMoMio}.}. Si vous prévoyez de n'y mettre que vos programmes, une taille entre 7 et 10 Gio sera amplement suffisante. Si vous souhaitez également y placer vos données personnelles, et donc ne pas créer de partition \Chemin{/home} comme décrit plus loin, 16 Gio constitueront un minimum.
\item La partition d'échange, appelée «~swap~» : 256 Mio est le minimum conseillé. La règle habituelle est de créer une partition swap égale au double de la taille de la \RefGlossaire{-}{RAM}{Mémoire vive}{mémoire vive}\NotePage{Cette mémoire vive est également appelée \RefGlossaire{-}{Mémoire vive}{RAM}{RAM}}. Ainsi, avec 512 Mio de \RefGlossaire{-}{Mémoire vive}{RAM}{RAM}, il faudrait créer une partition swap de 1Gio. Si vous manquez d'espace sur votre disque dur, une swap comprise entre 256 Mio et 1 Gio fera très bien son travail. Si vous souhaitez utiliser les fonctionnalités d'hibernation, il faut alors obligatoirement que la taille de la swap soit au moins égale à celle de la \RefGlossaire{-}{Mémoire vive}{RAM}{RAM}.
\item Enfin, la partition utilisateur nommée «~\Chemin{/home}~», n'est pas une nécessité\NotePage{D'ailleurs, le partitionnement automatique ne la crée pas}. Cependant, elle offre l'avantage de pouvoir partager le même utilisateur sur plusieurs systèmes GNU/Linux\NotePage{Notez que les développeurs ne garantissent généralement pas une compatibilité ascendante du format de vos données personnelles entre différentes versions d'un même logiciel. Par exemple, une mouture récente de Tomboy déplace vos données et il est alors impossible d'utiliser une version précédente avec celles-ci. Assurez-vous donc que les différents systèmes utilisent les mêmes versions de logiciels.} et de formater les autres partitions -- \Chemin{/} et swap -- sans avoir à sauvegarder préalablement les données des utilisateurs qui resteront «~à l'abri~» dans la partition \Chemin{/home}. Toutes vos données personnelles seront enregistrées sur cette partition. Par conséquent, dimensionnez-la en fonction de vos besoins.
\item Bien entendu, vous êtes libre de créer autant d'autres partitions que vous le souhaitez, si vous savez ce que vous faites.
\end{itemize}
\begin{nota}
FAT32 et NTFS sont des systèmes de fichiers utilisés par Windows. Un format de fichiers définit, en quelque sorte, la manière dont sont classées et enregistrées vos données sur le disque. Les partitions Linux utilisent, en général, le format EXT4 pour les données et les logiciels, ainsi que SWAP pour la partition swap. N'étant pas sectaire, Ubuntu peut aussi bien lire sur les partitions FAT32 que NTFS, alors que Windows ne sait pas lire et/ou écrire au format EXT4\NotePage{L'EXT4, qui est une évolution de l'EXT3 longtemps utilisé, apporte globalement une amélioration des performances.}\ldots{} Vous comprendrez donc rapidement qu'il vous faut utiliser les partitions en FAT32 ou en NTFS de Windows pour partager des données entre Ubuntu et Windows. Vous pouvez alors penser à ce stade que le plus simple consiste alors à mettre directement les partitions \Chemin{/home} et \Chemin{/} d'Ubuntu sur une partition au format FAT32 ou NTFS\dots{} Cela n'est pas conseillé car Ubuntu n'est pas optimisé pour ces types de partition et il préfère son propre système de fichiers\NotePage{Élémentaire mon cher Watson !}. De plus, si vous passez outre mes recommandations et mettez \Chemin{/home} et \Chemin{/} en FAT32 ou en NTFS, Windows va «~mettre le bazar\NotePage{Windows enregistre des fichiers cachés et fragmente énormément ses disques durs, ce que ne fait pas GNU/Linux}~» dans les données de GNU/Linux. Il faut donc, de préférence, que vous créiez une partition de partage séparée, en FAT32 ou en NTFS.
\end{nota}
\end{itemize}\par
Il vous faudra créer les partitions à l'aide de «~Ajouter\ldots{}~» et indiquer quelles partitions correspondent à «~\Chemin{/}~», «~swap~» et éventuellement les partitions supplémentaires décrites précédemment comme «~\Chemin{/home}~». Vous pouvez également cocher une case indiquant que vous désirez formater la partition --- opération obligatoire si vous venez de la créer. L'option «~Nouvelle table de partition\ldots{}~» vous permet d'effacer tout référencement à vos partitions sur votre disque dur. Par conséquent, ne l'utilisez qu'en sachant ce que vous faites. Pour plus d'informations sur les dossiers et la structure d'un système GNU/Linux, se référer à la section \ref{RefArboFichiers}.\par
Vous pouvez également redimensionner les partitions existantes en les éditant. Un clic sur «~Valider~» et un message vous demandera s'il faut appliquer immédiatement les modifications déjà effectuées. Il ne restera qu'à confirmer !
\begin{nota}
Avec le choix du partionnement manual, vous avez également la possibilité de choisir où installer GRUB -- le chargeur d'amorçage. Si vous ne savez pas ce que cela signifie et que votre PC n'est pas tatoué\NotePage{Se référer à l'encart sur le tatouage des PC, section \ref{RefTatouage}}, c'est que la valeur par défaut vous conviendra très bien !
\end{nota}
\subsubsection{Le mot de la fin}
Je rajoute enfin que malgré une idée très répandue, nul besoin de faire de partition \Chemin{/home} pour uniquement conserver ses données entre deux réinstallations -- si vous souhaitez réinstaller Ubuntu au lieu de faire une mise à jour. En effet, l'installeur d'ubuntu ne supprimera pas toutes les données de votre partition principale \Chemin{/} en cas de réinstallation sans cocher «~formater~» : il supprimera tout le système sauf vos données personnelles et quelques informations systèmes\NotePage{Quelques fichiers dans \Chemin{/etc} ainsi que vos compilations locales dans \Chemin{/usr/local}}. Encore mieux : dans le cas où vous utilisez les mêmes noms pour vos utilisateurs du système, il les retrouvera et appliquera les bonnes permissions afin que ces derniers réaccèdent à leurs fichiers sans aucune intervention de votre part. Comme quoi, un \Chemin{/home} séparé n'a vraiment pas lieu d'être.
\begin{nota}
Il sera possible d'accéder aux partitions des autres \RefGlossaire{Système}{OS}{Système d'exploitation}{systèmes d'exploitation} comme Windows, si vous les avez conservés, comme expliqué à la section \ref{AutrePartitions}.
\end{nota}
Dès que vous aurez fait votre choix, votre système commence à s'installer ! Pourquoi perdre du temps lorsque ce n'est pas nécessaire ? Cependant, il vous reste encore un peu de travail pendant qu'il copie et installe le système\ldots{}
\subsection{Emplacement géographique}
Choisissez ici votre fuseau horaire. Si ce dernier n'est pas le bon, cliquez sur la ville la plus proche de la carte comme Paris pour la France métropolitaine ou encore entrez le nom de ville dans le champ proposé. Si vous avez Windows sur le même ordinateur, je vous encourage à lire la section \ref{RefHeureNTP}.
\ImgCentree{8.78}{corps/chapitre2/img/Install5.png}{Choix du fuseau horaire}{ImgInstall5}
\subsection{Disposition du clavier}
Par défaut, la sélection se porte sur un type de clavier par défaut comme «~France - Autre~»\NotePage{En réalité, il s'agit d'une version dérivée du clavier fr-latin9} -- ce choix pouvant varier selon votre configuration, bien entendu -- qui correspond à un clavier français\NotePage{Clavier azerty} équipé de la touche «~\texteuro{}~» sur la touche E, ce qui devrait donc être la bonne disposition à moins que vous ne soyez Suisse ou Québécois. Si vous résidez en Belgique, vous devrez aussi porter votre choix sur ce clavier. Pour être sûr de ne pas vous tromper, vous pouvez tester la configuration choisie en tapant des caractères spéciaux dans l'espace prévu à cet effet en bas de la fenêtre.
\ImgCentree{8.78}{corps/chapitre2/img/Install6.png}{Choix du type de clavier}{ImgInstall6}
\begin{nota}
La disposition «~France - Autre~» permet entre autres de taper «~œ~» à la place de «~²~». Ces différences pourront s'avérer intéressantes pour certains.
\end{nota}
\subsection{Identité}
\AddIndex{-}{Login}{Identifiant}{-}
\label{InstallIdentite}
Il faut désormais indiquer votre nom, le nom que vous souhaitez donner à votre ordinateur, votre pseudonyme\NotePage{Surnom qui correspondra à votre \RefGlossaire{-}{Login}{Identifiant}{identifiant}}, et un mot de passe. Concernant le mot de passe, seules des étoiles apparaîtront à l'écran pour empêcher quiconque de le lire par dessus votre épaule. Pour être sûr que vous n'avez pas fait de fautes de frappe, il vous est demandé de le saisir une seconde fois. Une petite case vous permet de vous connecter automatiquement au démarrage de votre ordinateur. Ceci est conseillé si vous êtes l'unique utilisateur de votre ordinateur -- vous ne verrez pas alors l'écran d'accueil présenté à la section \ref{PremiereConnexion}, sinon, créez un compte par personne qui utilisera votre machine. Il s'agit là d'un des concepts forts des UNIX et dérivés et un avantage à l'usage. Alors, ne passez pas à côté de cette chance, vraiment. Vous noterez également qu'il est possible de chiffrer votre dossier personnel. Ceci est utile notamment sur les ordinateurs portables en cas de vol : votre mot de passe est nécessaire pour lire les données sur le disque dur, quelle que soit la méthode utilisée pour essayer d'en extraire les informations.\par
\ImgCentree{8}{corps/chapitre2/img/Nouvel_Utilisateur.png}{Création d'un nouvel utilisateur}{ImgNouvelUtilisateur}
\subsection{Migration des données depuis Windows vers Ubuntu}
Si votre disque dur comporte une version de Microsoft Windows, il vous est possible de récupérer certaines données depuis votre partition Windows comme vos favoris d'Internet Explorer, vos marques-pages Firefox, votre fond d'écran, vos contacts AOL et Yahoo par exemple. Il faudra alors lier ces paramètres à un compte. Les opérations à effectuer pour la création d'un compte sont similaires à celles de l'étape suivante. Si vous voulez ignorer cette étape, appuyez alors directement sur «~Suivant~».\par
\subsection{Et on attend!}
Il s'ensuit une étape où l'ensemble des sélections précédentes sont appliquées : création et initialisation des partitions, installation du système, configuration des langues, du clavier\ldots{} Comme indiqué auparavant, cette procédure a commencé dès que vous avez terminé le partitionnement. L'installation est donc d'une durée variable, dépendant donc du temps mis pour remplir les informations restantes, de la vitesse de votre lecteur CD-ROM, et des performances globales de votre ordinateur, mais on peut l'approximer à une moyenne de 10-15 minutes.\par
\ImgCentree{7}{corps/chapitre2/img/Installation_En_Cours.png}{Installation en cours, à peine le temps de prendre un café !}{ImgInstallInProgress}
Si votre connexion \RefGlossaire{-}{L'Internet}{Internet}{Internet} est active, les localisations françaises des logiciels seront téléchargées et installées tout comme les dernières mises à jour de sécurité. Ces opérations peuvent prendre du temps\NotePage{Cela est particulièrement le cas la première semaine de la sortie d'une nouvelle version}, notamment si votre connexion est lente ou les serveurs d'Ubuntu surchargés. Vous pouvez arrêter la procédure d'installation des traductions en cliquant à tout moment sur «~Ignorer~».\par
\ImgCentree{7}{corps/chapitre2/img/Telechargement_Paquets.png}{Installation de la localisation française}{ImgInstallFr}
Ceci terminé, vous aurez la possibilité de continuer à travailler sur votre session Live\NotePage{Attention, le système est toujours plus lent qu'un système installé et rien ne sera sauvegardé sur le disque dur}, ou de redémarrer -- avec demande de retrait du CD-ROM -- afin que votre ordinateur lance le nouveau système fraîchement installé. Il ne vous reste qu'à saisir le «~pseudo~» et le mot de passe que vous aviez fournis précédemment et\ldots{} en voiture\NotePage{Quoi ? Il ne s'agit pas d'automobile ? On m'aurait menti ? :-)} !!!\par
\ImgCentree{5}{corps/chapitre2/img/Installation_Terminee.png}{Redémarrer ?}{ImgInstall13}
Était-ce vraiment compliqué ? \ldots{}
